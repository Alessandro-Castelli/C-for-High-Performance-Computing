\documentclass[11pt]{article}

% Impostazioni del documento
\usepackage[a4paper,top=3cm,bottom=3cm,left=2cm,right=2cm,marginparwidth=2.5cm]{geometry}
\usepackage{parskip} % Spazio tra i paragrafi
\usepackage{setspace} % Interlinea
\onehalfspacing % Interlinea 1.5

% Impostazioni del font
\usepackage{mathptmx} % Times New Roman
\usepackage[T1]{fontenc}

% Altri
\usepackage{multicol}
\usepackage{graphicx}
\usepackage{enumitem}
\usepackage{hyperref}
\hypersetup{
    colorlinks=true,    % Rendi i link colorati
    linkcolor=blue,     % Colore dei link interni (es. sezioni)
    citecolor=blue,     % Colore delle citazioni bibliografiche
    urlcolor=blue       % Colore degli URL
}

\begin{document}

\begin{titlepage}
    \centering
    \includegraphics[width=0.5\textwidth]{aau.png}
    \vfill
    \begin{center}
        {\LARGE C++ for High-Performance Computing \par}
        \vspace{0.5cm}
        {\large Alessandro Castelli [12246581] \par}
        \vspace{0.5cm}
        {\large \today \par}
    \end{center}
    \vfill
\end{titlepage}

\begin{multicols*}{2}[\columnsep=1cm]
    
    \section{Introduction}
    Nel seguente elaborato saranno analizzati e discussi tre paper che parlano di come può essere usato il linguaggio di programmazione \texttt{C$++$} per il calcolo ad alte prestazioni.
    Sono stati individuati tre paper scientifici che affrontano questo tema da diversi punti di vista.
    I tre paper individuati sono i seguenti:
    \begin{enumerate}
        \item \textit{C++ Reflection for High Performance Problem Solving Environments} \cite{Article1}, che è attualmente reperibile al seguente link \href{https://citeseerx.ist.psu.edu/document?repid=rep1&type=pdf&doi=2058bb40e6b80504ba1084452fd9c126cd19f891}{https://citeseerx.ist.psu.edu}
        \item \textit{How templates enable High-Performance Scientifing Computing in in C++} \cite{Article2}, che è attualmente reperibile al seguente link \href{https://ieeexplore.ieee.org/abstract/document/774843?casa_token=YqZfo7t12KoAAAAA:aUt-msPVNEAtzfVwO4h_-R_r7IPTFs7vHYHbAtsOdDE83PlNvB8gkNl5maWpHYBU5QkS3cUp0R8}{https://ieeexplore.ieee.org}
        \item \textit{Conduit: A C++ Library for Best-effort High Performance Computing} \cite{Article3}, che è attualmente reperibile al seguente link \href{https://dl.acm.org/doi/abs/10.1145/3449726.3463205}{https://dl.acm.org}
    \end{enumerate}


\end{multicols*}

\bibliographystyle{plain}
\bibliography{Bib.bib}

\end{document}
